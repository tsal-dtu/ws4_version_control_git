%\documentclass[10pt,a4paper,final]{article}
\documentclass[10pt,a4paper]{article}
\usepackage[utf8]{inputenc}
\usepackage{amsmath}
\usepackage{amsfonts}
\usepackage{amssymb}
%\usepackage[pdftex]{hyperref}
\usepackage{hyperref}
\usepackage{listings}

\usepackage{color}
 
\definecolor{codegreen}{rgb}{0,0.6,0}
\definecolor{codegray}{rgb}{0.5,0.5,0.5}
\definecolor{codepurple}{rgb}{0.58,0,0.82}
\definecolor{backcolour}{rgb}{0.95,0.95,0.92}
 
\lstdefinestyle{mystyle}{
    backgroundcolor=\color{backcolour},   
    commentstyle=\color{codegreen},
    keywordstyle=\color{magenta},
%    numberstyle=\tiny\color{codegray},
    stringstyle=\color{codepurple},
    basicstyle=\footnotesize,
    breakatwhitespace=false,         
    breaklines=true,                 
    captionpos=b,                    
    keepspaces=true,                 
%    numbers=left,                    
    numbersep=5pt,                  
    showspaces=false,                
    showstringspaces=false,
    showtabs=false,                  
    tabsize=2
}
 
\lstset{style=mystyle}


\title{Version control using Git}
\author{Tommy S. Alstrøm \\\\
Document version 0.1\\\\
November 16, 2015}
\date{\vspace{-5ex}}

\begin{document}
\maketitle
\section{Introduction}
This document is a draft that contains condensed information about how to use GIT.


\section{Installing git}



\section{Getting started}
On DTU compute, we have two git servers. The first runs the clasical 

\begin{lstlisting}
# git init
Initialized empty Git repository in <path>
\end{lstlisting}
This command creates the .git directory which contains the ``system files'' of git.

Adding files is done with
\begin{lstlisting}
# git add
Initialized empty Git repository in <path>
\end{lstlisting}


\section{Uploading your git repo to a server}


\section{Branching}


\section{Installing GNU Make on Windows}
The example depot contains make files. Make is not available on windows out of 
the box and must be downloaded. It can be fetched here:\\ 
\url{http://www.mingw.org/wiki/Getting_Started}\\ Download the tool called 
\texttt{mingw-get-setup.exe} and choose the package called \texttt{msys-base}. 
This will install the necessary tools in 
\texttt{C:\textbackslash{}MinGW\textbackslash{}msys\textbackslash{}1.0}. A bash 
shell can now be started by running 
\texttt{C:\textbackslash{}MinGW\textbackslash{}msys\textbackslash{} 
1.0\textbackslash{}msys.bat}. All drives on the computer is automatically 
mounted in root, so e.g. to access the C-drive type \texttt{cd /c}.

\section{Interesting links}
Linus (the creator of Git) talks about git and large files:\\
\url{http://osdir.com/ml/git/2009-05/msg00051.html}
\end{document}